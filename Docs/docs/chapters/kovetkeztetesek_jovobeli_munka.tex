Bár a drón jelenleg még nem képes a felszállásra, a projekt jelentős előrelépést jelentett a tervezés, szimuláció és kivitelezés területén. A következő lépések közé tartozik:

A szimulációs környezet pontosítása a valós repülési viselkedés jobb megközelítése érdekében.

A drón összeszerelésének befejezése és a mérési eredmények alapján történő optimalizáció.

Az autonóm irányítási algoritmus fejlesztése és implementálása.

A mesterséges intelligencia alkalmazása a rendszerben, amely lehetővé teszi a drón intelligensebb, adaptív irányítását és képes lehet önálló döntéshozatalra is. Ez nemcsak technológiai előrelépést jelentene, hanem trendi megoldásként a projekt piaci értékét is növelné.

A projekt hátralevő részei izgalmas lehetőségeket tartogatnak a további tanulmányok és kísérletek számára, amelyek hozzájárulhatnak a dróntechnológia átfogó megértéséhez.