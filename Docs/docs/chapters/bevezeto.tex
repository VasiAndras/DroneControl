Az utóbbi évtizedek technológiai fejlődése alapvetően megváltoztatta mindennapjainkat, különösen az autonóm járművek és a drónok területén. A drónok, vagy más néven pilóta nélküli légijárművek (UAV-k), egyre elterjedtebbé váltak különféle iparágakban és alkalmazásokban. A mezőgazdaságtól kezdve a katasztrófavédelemig, a filmipartól a haditechnikáig számos területen használják őket hatékonyságuk és sokoldalúságuk miatt.

A drónok irányítása és vezérlése az egyik legizgalmasabb és leginnovatívabb kutatási terület, amely számos tudományágat ötvöz, mint például a robotika, az elektronika, a mesterséges intelligencia és a távközlés. Ahhoz, hogy egy drón képes legyen irányított vagy autonóm módon repülni és különböző feladatokat ellátni, fejlett algoritmusokra és érzékelőkre van szükség. Ezen rendszerek fejlesztése és integrációja komoly mérnöki kihívást jelent, ugyanakkor lehetőséget kínál új technológiák és megoldások kifejlesztésére.

Projekten célja egy saját tervezésű drón létrehozása, amely képes távirányított, majd későbbiekben akár autonóm módon repülni, navigálni és különböző feladatokat elvégezni. A drón építése és fejlesztése során a legújabb technológiákat és módszereket alkalmaztam, mint például a drón 3D fizikai modellezése, 3D kivitelezése (nyomtatása 3D nyomtatóval), irányítási algoritmusok szimulálása MATLAB környezetben. Terveim szerint, ezekkel a technológiákkal egy olyan rendszert hozhatunk létre, amely megbízható és hatékony a valós környezetben történő alkalmazás során. A projekt során különös figyelmet fordítottam a drón irányítási rendszerének kidolgozására és szimulációjára, amely magában foglalja az érzékelők adatainak feldolgozását, a navigációs algoritmusok fejlesztését, valamint a kommunikációs és vezérlő rendszerek integrációját.

A projekt egyik fő kihívása az, hogy olyan irányítási rendszert hozzak létre, amely képes közel valós időben reagálni a környezeti változásokra és az előre nem látható helyzetekre. 
