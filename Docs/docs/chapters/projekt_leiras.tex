Jelenleg egy saját fejlesztésű drón építésén dolgozom, ezen egyetemi tantárgy keretében valósul meg. A drón fizikailag már megépült nagyjából, a konstrukcióhoz 3D nyomtatott technológia került alkalmazásra. A fejlesztés előzménye szakirodalmi kutatás volt, mely során különböző dróntechnológiák és szabályozási módszerek kerültek áttekintésre. A drón szimulációja MATLAB Simulink környezetben zajlik, amely lehetővé teszi a különböző rendszerek és vezérlési algoritmusok modellezését.

Bár a drón még nem készült el teljesen, a projekt folyamatban van, és az építés előrehaladott állapotban van. A drón jelenlegi formája még további finomításra szorul, és számos lehetőség rejlik a fejlesztésében. Az aktuális verzióval kapcsolatosan több szempontból is van lehetőség a teljesítmény javítására, különösen a vezérlőrendszerek optimalizálása, az akkumulátor élettartamának növelése és az irányíthatóság fokozása terén.

A drón befejezése várhatóan a közeljövőben megtörténik, és az egyetemi projekt célja, hogy lehetőséget adjon a valós idejű alkalmazások tesztelésére, valamint további fejlesztési irányok feltárására is. A projekt nemcsak technológiai kihívásokat tartogat, hanem számos lehetőséget is biztosít a drónokkal kapcsolatos kutatások és alkalmazások mélyebb megértésére.