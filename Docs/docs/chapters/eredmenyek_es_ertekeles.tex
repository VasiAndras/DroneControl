Az egyetemi tárgy keretében olyan projektet választottam, amely a drónok irányítását és fejlesztését állítja a középpontba. A projekt célja egy saját tervezésű és készítésű drón megépítése, szimulációs környezetben történő tesztelése, valamint a repülési képességek és az autonóm irányítás alapjainak kidolgozása. Ez a fejezet a projekt jelenlegi állapotát, a megvalósított részleteket és a felmerült kihívásokat ismerteti.

A drón majdnem teljes egészében saját tervezésű, a fő szerkezeti elemek 3D nyomtatással készültek. A kivitelezés során nagy hangsúly került a stabilitásra és a szerkezeti integritásra, mivel ezek alapvetõen befolyásolják a drón repülési tulajdonságait. Az elektronikai komponensek közül az irányítási rendszert és a szenzorokat a projekt összetettségének megfelelően választottam ki. A drón jelenleg még nem képes felszállni, de a szerkezeti és műszaki elemek nagy része már elkészült, és az összeszerelés utolsó fázisában van.

A projekt szerves részét képezi a szimulációs környezet kialakítása, amelyben a drón repülési képességeit és irányítási algoritmusait lehet tesztelni. A MATLAB Simulink használatával igyekeztem modellezni a drón fizikai rendszerét és a környezeti hatásokat. A szimuláció közben azonban számos kihívással szembesültem:

-  rendszer pontossága: A drón dinamikájának modellezése bonyolultabbnak bizonyult, mint azt előzetesen feltételeztem, ami némi eltérést eredményezett a valós viselkedéshez képest.

- Környezeti modellezés: A külső hatások, mint például a szél, turbulencia és gravitáció szimulációja külső források és referenciaanyagok bevonását igényelte.

Ezek a nehezíségek ellenére sikerült létrehozni egy alapvetően működő rendszert, amely alkalmas a további fejlesztések alapjainak lefektetésére.
