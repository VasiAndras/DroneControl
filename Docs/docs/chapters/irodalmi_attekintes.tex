Az utóbbi években a drónok tervezésének területe jelentős fejlődésen ment keresztül, amelyet a kutatások és technológiai innovációk növekvő mennyisége és minősége táplált. Az átfogó szakirodalmi áttekintések sokoldalú alkalmazásokat és technológiai fejlesztéseket tártak fel.

Például a megfigyelési célú drónokkal kapcsolatos kutatások kiemelik a nehezebb-légkörű (Heavier-Than-Air, HTA) és a könnyebb-légkörű (Lighter-Than-Air, LTA) UAV-rendszerek közötti különbségeket. Az ilyen kutatások egyre inkább a kettős rendszerek alkalmazását javasolják, amelyek ötvözik mindkét kategória előnyeit, például a HTA drónok mozgékonyságát és az LTA rendszerek energiahatékonyságát.

További szisztematikus irodalom-áttekintések hangsúlyozzák a dróntechnológia folyamatos fejlődését az elmúlt évtized során. Ezek az áttekintések nem csupán az új technológiák bevezetéséről számolnak be, hanem a drónok alkalmazási körének folyamatos bővüléséről is, különös tekintettel az ipari, mezőgazdasági, logisztikai és környezetvédelmi területekre.

Külön figyelmet érdemelnek a drónok logisztikai alkalmazásairól szóló kutatások, amelyek rávilágítanak ezen eszközök lehetőségeire az árukészlet-ellenőrzés, az utolsó kilométeres kiszállítás és a fenntarthatóság terén. Ezek a tanulmányok alátámasztják, hogy a drónok hatékonyan támogathatják az ellátási lánc optimalizálását, miközben csökkentik a környezeti terhelést, például az üvegházhatású gázok kibocsátását.

E kutatások összességében a dróntervezés dinamikus és folyamatosan fejlődő természetét tükrözik, előkészítve az utat a jövőbeni innovációk és gyakorlati alkalmazások számára. A különböző fejlesztési irányok, mint például az autonóm navigáció, az energiatárolási technológiák javítása és az érzékelőrendszerek integrációja, tovább fokozzák a drónok potenciálját, hogy alapvető szereplőkké váljanak a modern ipari és társadalmi környezetben.
\cite{tadic2021application}, \cite{doornbos2024drone}, \cite{adorni2021literature}

\section{Szabályozási algoritmusok áttekintése: Hagyományos és modern megközelítések}

A szabályozási algoritmusok kulcsszerepet töltenek be számos mérnöki és technológiai alkalmazásban, mivel hatékony eszközt nyújtanak rendszerek és folyamatok irányításához. Az egyik legszélesebb körben alkalmazott szabályozási algoritmus a Proporcionális-Integráló-Deriváló (PID) szabályozó. A PID szabályozók egyszerűségükről és sokoldalúságukról ismertek, legyen szó ipari automatizálásról vagy robotikáról. Ezek az algoritmusok a kívánt érték (setpoint) és az aktuális folyamatváltozó közötti eltérés alapján állítják be a szabályozási jelet, a proporcionális, integráló és deriváló tagok segítségével minimalizálva az eltérést. A PID szabályozók hangolása széles körben kutatott terület, számos módszert dolgoztak ki a teljesítményük optimalizálására1.

Egy másik jelentős szabályozási algoritmus a modell-alapú prediktív szabályozás (Model Predictive Control, MPC), amely népszerűsége a többváltozós rendszerek és a korlátozások kezelésére való képességéből fakad. Az MPC egy dinamikus folyamatmodellt használ a rendszer jövőbeni viselkedésének előrejelzésére, és minden szabályozási lépésben egy optimalizálási problémát old meg az optimális szabályozási akciók meghatározására. Ez az előrelátó megközelítés lehetővé teszi az MPC számára, hogy proaktívan reagáljon jövőbeli eseményekre, így különösen alkalmas komplex ipari folyamatok és fejlett gyártórendszerek szabályozására2. Az MPC rugalmassága és robusztussága miatt széles körben alkalmazzák többek között vegyészmérnöki, autóipari és repülőgépipari rendszerekben.

Az intelligens szabályozási algoritmusok, például a fuzzy logikán, neurális hálózatokon és genetikus algoritmusokon alapuló megoldások szintén jelentős fejlődésen mentek keresztül. Ezek az algoritmusok a biológiai folyamatokat utánozva képesek kezelni a bizonytalanságokat és a nemlinearitásokat a szabályozási rendszerekben. Például a fuzzy logikán alapuló szabályozók nyelvi szabályokkal modellezik a komplex rendszereket, ami különösen hasznos azokban az alkalmazásokban, ahol a pontos matematikai modellek nehezen elérhetők3. A neurális hálózatok ezzel szemben képesek adatokból tanulni és alkalmazkodni a változó körülményekhez, így erőteljes eszközt kínálnak az adaptív szabályozás számára. A genetikus algoritmusok a természetes szelekció folyamatát szimulálva optimalizálják a szabályozási paramétereket3.

A hagyományos és intelligens szabályozási algoritmusok mellett egyre nagyobb figyelmet kapnak a hibrid szabályozási rendszerek, amelyek több szabályozási stratégia kombinációját használják azok előnyeinek kihasználására. Például a PID szabályozók és a fuzzy logika kombinálása javíthatja a szabályozási rendszer teljesítményét, egyszerre biztosítva precíz szabályozást és robusztusságot a bizonytalanságokkal szemben. Hasonlóképpen, az MPC és a neurális hálózatok integrációja növelheti a szabályozási rendszer előrejelzési pontosságát és alkalmazkodóképességét2. Ezek a hibrid megközelítések különösen hasznosak komplex és dinamikus környezetekben, ahol egyetlen szabályozási stratégia nem elegendő.

A szabályozási algoritmusok folyamatos fejlődését a számítástechnika teljesítményének növekedése és a modern rendszerek növekvő komplexitása hajtja. Ahogy az iparágak a smart manufacturing és az Ipar 4.0 irányába haladnak, egyre nagyobb az igény olyan fejlett szabályozási algoritmusokra, amelyek képesek nagy léptékű, összekapcsolt rendszereket kezelni. A kutatások célja hatékonyabb és skálázhatóbb algoritmusok fejlesztése, valamint új alkalmazási területek feltárása az autonóm járművek, a megújuló energiaforrások és a Dolgok Internete (IoT) terén. A jövő szabályozási algoritmusai várhatóan képesek lesznek alkalmazkodni a változó körülményekhez, tanulni az adatokból, és valós időben biztosítani az optimális teljesítményt.

\cite{lee2023review}, \cite{schwenzer2021review}, \cite{borase2021review}